%%%%%%%%%%%%%%%%%%%%%%%%%%%%%%%%%%%%%%%%%%%%%%%%%%%%%%%%%%%%%%%%%%%%%%%%%%%%%%%%%%%%%%%%%
% Appendix C: DeviceInfo XML Files
%	This section provides a detailed description of Family Info files, and their
%   syntax. Most users of RapidSmith2 don't need to know the syntax of family info
%   files, but this information may be useful to new lab members working on the
%   project.
%%%%%%%%%%%%%%%%%%%%%%%%%%%%%%%%%%%%%%%%%%%%%%%%%%%%%%%%%%%%%%%%%%%%%%%%%%%%%%%%%%%%%%%%%
\newpage
\section{DeviceInfo Info XML} \label{sec:deviceInfo}
A device info XML file contains additional information \textbf{specific to a
device} that is not found in the corresponding XDLRC for the device. Currently,
the device info file contains only a list of package pins for the device as
shown in \autoref{lst:deviceInfoXml}. Each package pin definition has three
attributes:

\begin{enumerate}
  \item \textbf{The name of the package pin}: Generally, package pin names are a
  single letter followed by a two-digit number (i.e. M17). For those that have
  written UCF or XDC constraints for a FPGA design targeting a Xilinx part, this
  format should be familiar.
  \item \textbf{The corresponding PAD BEL for the package pin}: Each
  package pin maps to a specific BEL in the device. Both the name of the
  BEL as well as its parent site is recorded in the form ``site/belname.''
  \item \textbf{An optional ``is\_clock'' attribute}: Only a select number of
  package pins in a device can access the global clock routing resources. These
  package pins are explicitly marked in the device info file so external CAD
  tools can use this information when placing clock ports (or other signals
  that need access to global routing).
\end{enumerate}

\noindent Device info files can be generated with the \texttt{Tincr}
command \texttt{[tincr::create\_xml\_device\-\_info]}. This command is fully
automated, and requires no hand edits.

\begin{lstlisting}[numbers=none, keywordstyle=, stringstyle=,
caption=Example Device Info XML, label=lst:deviceInfoXml]
  <device_info>
      <partname>xcku025-ffva1156</partname>
      <package_pins>
          <package_pin>
              <name>AK33</name>
              <bel>IOB_X0Y155/PAD</bel>
          </package_pin>
          <package_pin>
              <name>AJ29</name>
              <bel>IOB_X0Y130/PAD</bel>
              <is_clock/>
          </package_pin>
          ...
      </package_pins>
  </device_info>
\end{lstlisting}
