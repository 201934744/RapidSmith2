%%%%%%%%%%%%%%%%%%%%%%%%%%%%%%%%%%%%%%%%%%%%%%%%%%%%%%%%%%%%%%%%%%%%%%%%%%%%%%%%%%%%%%%%%
% Section 13: Dependency Projects
%	This section contains a list of software dependencies in order to run RapidSmith2.
%	It is important to note that these dependencies are included in the repository
% 	so users don't have to worry about downloading the dependencies themselves.
%%%%%%%%%%%%%%%%%%%%%%%%%%%%%%%%%%%%%%%%%%%%%%%%%%%%%%%%%%%%%%%%%%%%%%%%%%%%%%%%%%%%%%%%%
\newpage
\section{Included Dependency Projects}
RapidSmith2 includes the Caucho Technology Hessian implementation which is distributed
under the Apache License. A copy of this license is included in the doc
directory in the file APACHE2-LICENSE.txt. This license is also available for
download at:

\noindent
\hyperref[http://www.apache.org/licenses/LICENSE-2.0]{\color{blue}http://www.apache.org/licenses/LICENSE-2.0}

\bigbreak \noindent
The source for the Caucho Technology Hessian implementation is available at:

\noindent
\hyperref[http://hessian.caucho.com]{\color{blue}http://hessian.caucho.com}

\bigbreak \noindent
RapidSmith2 also includes the Qt Jambi project jars for Windows, Linux and Mac OS X.  Qt
Jambi is distributed under the LGPL GPL3 license and copies of this license and
exception are also available in the /doc directory in files LICENSE.GPL3.TXT and
LICENSE.LGPL.TXT respectively. These licenses can also be downloaded at:

\noindent
\hyperref[http://www.gnu.org/licenses/licenses.html]{\color{blue}http://www.gnu.org/licenses/licenses.html}

\bigbreak \noindent
Source for the Qt Jambi project is available at:

\noindent
\hyperref[http://qtjambi.org/downloads]{\color{blue}http://qtjambi.org/downloads}, and

\noindent
\hyperref[https://sourceforge.net/projects/qtjambi/files/]{\color{blue}https://sourceforge.net/projects/qtjambi/files/}

\bigbreak \noindent
RapidSmith2 also includes the JOpt Simple option parser which is released under
the open source MIT License which can be found in this directory in the file
MIT\_LICENSE.TXT.  A copy of this license can also be found at:

\noindent
\hyperref[http://www.opensource.org/licenses/mit-license.php]{\color{blue}http://www.opensource.org/licenses/mit-license.php}

\bigbreak \noindent   
A copy of the source for JOpt Simple can also be downloaded at:

\noindent
\hyperref[http://jopt-simple.sourceforge.net/download.html]{\color{blue}http://jopt-simple.sourceforge.net/download.html}

\bigbreak \noindent
RapidSmith2 also includes the JDOM jars.  JDOM is available under an Apache-style open
source license, with the acknowledgment clause removed. This license is among
the least restrictive license available, enabling developers to use JDOM in
creating new products without requiring them to release their own products as
open source. This is the license model used by the Apache Project, which created
the Apache server. The license is available at the top of every source file and
in LICENSE.txt in the root of the JDOM distribution.              

\bigbreak \noindent
The user is responsible for providing copies of these licenses and making
available the source code of these projects when redistributing these jars.