%%%%%%%%%%%%%%%%%%%%%%%%%%%%%%%%%%%%%%%%%%%%%%%%%%%%%%%%%%%%%%%%
% Section 1: Introduction
%%%%%%%%%%%%%%%%%%%%%%%%%%%%%%%%%%%%%%%%%%%%%%%%%%%%%%%%%%%%%%%%
\newpage
\section{Introduction}
\subsection{What is RapidSmith 2?}
The original BYU RapidSmith project began in 2010. Its goal was to develop
a set of tools and APIs which would provide academics with an
easy-to-use platform to implement experimental CAD ideas and algorithms on
modern Xilinx FPGAs. It integrated with Xilinx's old design suite, ISE.
RapidSmith 2 (abbreviated RS2 hereafter) represents a major addition to
RapidSmith. Specifically, Vivado designs are now supported. Using RS2 you can
write custom CAD tools which will: 
\begin{itemize}
  \item Export designs from Vivado
  \item Perform analyses on those designs
  \item Make modifications to those designs
  \item Import those designs back into Vivado for further processing or
  bitstream generation
\end{itemize}
Futhermore, you need not start with a Vivado design --- 
you can create a new design from scratch in RS2 and then import it into Vivado
if desired.

The other new major capability of RS2 is that it changes RapidSmith's design
representation. Instead of using XDL's view of a design with Instances and
Sites, RS2 uses Vivado's representation of design with Cells and BELs. This
is a significant change as it exposes the actual design and device in a way
that RapidSmith never did, opening up new CAD research opportunities which were
difficult to perform using Rapidsmith.
       
\subsection{Who Should Use RS2?}
RS2 is aimed at anyone desiring to do FPGA CAD research on real Xilinx devices
available in Vivado. As such, users of RS2 should have some understanding of
Xilinx FPGA architecture, the Vivado design suite, and the Tcl programming
language. However, one goal of this documentation is to provide sufficient
background and detail to help bring developers up to speed on the needed
topics. RS2 is by no means a Xilinx Vivado replacement. It cannot be used
without a valid and current license to a Vivado installation (RS2
cannot generate bitstreams for example).

\subsection{Why RS2?}
The Xilinx-provided Tcl interface into Vivado is a great addition to the tool
suite. It can be used to do a variety of useful things including scripting
design flows, querying device and design data structures, and  modifying placed
and routed designs. In theory, the Tcl interface provides all of the
functionality needed in order to create any type of CAD tool as a plugin to the
normal Vivado tool flow. However, there are a few issues in TCL that motivate
the use of external CAD tool frameworks such as RS2. These include:
\begin{itemize}
  \item Tcl, being an interpreted language, is slow. It is far too slow to
  implement complex algorithms such as PathFinder. Compiled and
  managed runtime languages are a better option in terms of performance.
  \item Tcl is hard to program in. TCL is not an object oriented language, and
  so writing complex algorithms are difficult since Object-Oriented language
  constructs do not exist. That being said, TCL is great for writing automation
  scripts.
  \item There are some memory issues in Vivado's Tcl interface. In our
  experience, long-running scripts eventually cause the system to run out of
  memory even if they are not doing anything interesting.
  \item Vivado's TCL interface does not offer a complete device representation
  (determined by Brad White's MS work). Most notably, a user cannot gain
  access to sub-site wire objects through the Tcl interface. This limits the CAD
  tools that can be created in Tcl, but this additional information can be added
  to external tools with some manual work.
\end{itemize}

\noindent
In short, the ability to export designs out of Vivado, manipulate them with more
powerful languages such as Java, and then import the design back into Vivado
is a very useful capability.

RS2 (in conjunction with Tincr which is described in \autoref{sec:tincr})
abstracts this process into a few easy-to-use function calls. Generating FPGA
part information, importing and exporting \pgm{all aspects} of a design, and
dealing with other fairly arcane details is made mostly transparent to the
user. RS2 and Tincr provide a nice API into equivalent Vivado device and design
data structures. All of this enables researchers to have more time to focus on what matters
most: the research of new ideas and algorithms.

\subsection{Which Xilinx Parts does RS2 Support?}
As of the writing of this document, Artix 7 has been tested the most and is
currently supported in all forms and applications.  In addition, an Ultrascale
device file was created and demonstrated as a part of Brad White's MS work to
show that it is possible. At some point, Ultrascale should be fully supported.
\footnote{An XDL-based import/export capability has also been created and used
with Virtex 6 devices as a part of Travis Haroldsen's PhD work but that path is
not being released, documented, or supported.}

As will be seen later, to generate additional device files for additional parts
within a supported family is relatively straightforward and can be done by any
user.  New families can also be supported but this
requires a bit more work.  As time goes on the process will become simpler ---
that is one of the goals for RS2 moving forward.

\subsection{How is RS2 Different than VPR and VTR?}
VPR (Versatile Place and Route) has been an FPGA research tool for several years
and has led to many publications on new FPGA CAD research. It has been a
significant contribution to the FPGA research community and has grown to be a
complete FPGA CAD flow for research-based FPGAs. The main difference between
RapidSmith and VPR is that the RapidSmith tools can target commercial Xilinx
FPGAs, providing the ability to exit and re-enter the standard Xilinx flow at
any point.  All features of commercial FPGAs which are accessible via XDL and
Vivado's Tcl interface are available in RapidSmith and RS2. VPR is currently
limited to FPGA features which can be described using VPR's architectural
description facilities.

\subsection{Why Java?}
RS2 is written in Java. We have found Java to be an excellent rapid prototyping
platform for FPGA CAD tools.  Java libraries are rich with useful data
structures, and garbage collection eliminates the need to clean up objects in
memory. This helps reduce the time spent debugging, leaving more
time for researchers to focus on the real research at hand.  Our experience over
the past decade is that for student research projects, Java has greatly improved
student productivity and led to far more stable CAD tools.