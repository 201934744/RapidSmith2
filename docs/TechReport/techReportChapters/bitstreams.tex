%%%%%%%%%%%%%%%%%%%%%%%%%%%%%%%%%%%%%%%%%%%%%%%%%%%%%%%%%%%%%%%%%%%%%%%%%%%%%%%%%%%%%%%%%
% Section 11: Bitstreams in RapidSmith2
%	This section describes how bitstreams are handled in RapidSmith2 (they currently
%   aren't)
%%%%%%%%%%%%%%%%%%%%%%%%%%%%%%%%%%%%%%%%%%%%%%%%%%%%%%%%%%%%%%%%%%%%%%%%%%%%%%%%%%%%%%%%%
\newpage
\section{Bitstreams in RapidSmith2}

In the original RapidSmith, bitstreams can be parsed, manipulated, and exported
for Virtex 4, Virtex 5 and Virtex 6 Xilinx FPGA families.  Because of the
proprietary nature of Xilinx bitstreams, RapidSmith provided only documented
functionality when working with bitstreams (and was limited mainly to
manipulation at the frame level including helping to assemble sequences of
configuration commands which are interpreted by the FPGA configuration
controller circuitry).  While this has proven valuable to many researchers, it
does not provide the ability to create your own bitstream from scratch because
it does not provide the specific meaning of each bit in a bitstream.

If you desire to use RapidSmith's bitstream manipulation features, you should
download and work with RapidSmith instead of RapidSmith2 (the RapidSmith bitstream
packages have been removed from RapidSmith2).  If you do so, note that RapidSmith's
bitstream packages have not been tested beyond Virtex 6.  The authors would be
interested in upgrading RapidSmith's bitstream functionality to device families
beyond Virtex 6 if users create it and are willing to contribute it to us for
inclusion.